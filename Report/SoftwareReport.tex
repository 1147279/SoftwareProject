\documentclass[11pt,journal, a4paper]{IEEEtran}

\makeatletter
\newcommand\subparagraph{%
  \@startsection{subparagraph}{5}
  {\parindent}
  {3.25ex \@plus 1ex \@minus .2ex}
  {-1em}
  {\normalfont\normalsize\bfseries}}
\makeatother
\usepackage{titlesec}
\let\subparagraph\relax % You don't want to use \subparagraph

% some very useful LaTeX packages include:

%\usepackage{cite}      % Written by Donald Arseneau
                        % V1.6 and later of IEEEtran pre-defines the format
                        % of the cite.sty package \cite{} output to follow
                        % that of IEEE. Loading the cite package will
                        % result in citation numbers being automatically
                        % sorted and properly "ranged". i.e.,
                        % [1], [9], [2], [7], [5], [6]
                        % (without using cite.sty)
                        % will become:
                        % [1], [2], [5]--[7], [9] (using cite.sty)
                        % cite.sty's \cite will automatically add leading
                        % space, if needed. Use cite.sty's noadjust option
                        % (cite.sty V3.8 and later) if you want to turn this
                        % off. cite.sty is already installed on most LaTeX
                        % systems. The latest version can be obtained at:
                        % http://www.ctan.org/tex-archive/macros/latex/contrib/supported/cite/
\usepackage{float}
\usepackage{graphicx}   % Written by David Carlisle and Sebastian Rahtz
                        % Required if you want graphics, photos, etc.
                        % graphicx.sty is already installed on most LaTeX
                        % systems. The latest version and documentation can
                        % be obtained at:
                        % http://www.ctan.org/tex-archive/macros/latex/required/graphics/
                        % Another good source of documentation is "Using
                        % Imported Graphics in LaTeX2e" by Keith Reckdahl
                        % which can be found as esplatex.ps and epslatex.pdf
                        % at: http://www.ctan.org/tex-archive/info/

%\usepackage{psfrag}    % Written by Craig Barratt, Michael C. Grant,
                        % and David Carlisle
                        % This package allows you to substitute LaTeX
                        % commands for text in imported EPS graphic files.
                        % In this way, LaTeX symbols can be placed into
                        % graphics that have been generated by other
                        % applications. You must use latex->dvips->ps2pdf
                        % workflow (not direct pdf output from pdflatex) if
                        % you wish to use this capability because it works
                        % via some PostScript tricks. Alternatively, the
                        % graphics could be processed as separate files via
                        % psfrag and dvips, then converted to PDF for
                        % inclusion in the main file which uses pdflatex.
                        % Docs are in "The PSfrag System" by Michael C. Grant
                        % and David Carlisle. There is also some information
                        % about using psfrag in "Using Imported Graphics in
                        % LaTeX2e" by Keith Reckdahl which documents the
                        % graphicx package (see above). The psfrag package
                        % and documentation can be obtained at:
                        % http://www.ctan.org/tex-archive/macros/latex/contrib/supported/psfrag/

%\usepackage{subfigure} % Written by Steven Douglas Cochran
                        % This package makes it easy to put subfigures
                        % in your figures. i.e., "figure 1a and 1b"
                        % Docs are in "Using Imported Graphics in LaTeX2e"
                        % by Keith Reckdahl which also documents the graphicx
                        % package (see above). subfigure.sty is already
                        % installed on most LaTeX systems. The latest version
                        % and documentation can be obtained at:
                        % http://www.ctan.org/tex-archive/macros/latex/contrib/supported/subfigure/

\usepackage{url}        % Written by Donald Arseneau
                        % Provides better support for handling and breaking
                        % URLs. url.sty is already installed on most LaTeX
                        % systems. The latest version can be obtained at:
                        % http://www.ctan.org/tex-archive/macros/latex/contrib/other/misc/
                        % Read the url.sty source comments for usage information.

%\usepackage{stfloats}  % Written by Sigitas Tolusis
                        % Gives LaTeX2e the ability to do double column
                        % floats at the bottom of the page as well as the top.
                        % (e.g., "\begin{figure*}[!b]" is not normally
                        % possible in LaTeX2e). This is an invasive package
                        % which rewrites many portions of the LaTeX2e output
                        % routines. It may not work with other packages that
                        % modify the LaTeX2e output routine and/or with other
                        % versions of LaTeX. The latest version and
                        % documentation can be obtained at:
                        % http://www.ctan.org/tex-archive/macros/latex/contrib/supported/sttools/
                        % Documentation is contained in the stfloats.sty
                        % comments as well as in the presfull.pdf file.
                        % Do not use the stfloats baselinefloat ability as
                        % IEEE does not allow \baselineskip to stretch.
                        % Authors submitting work to the IEEE should note
                        % that IEEE rarely uses double column equations and
                        % that authors should try to avoid such use.
                        % Do not be tempted to use the cuted.sty or
                        % midfloat.sty package (by the same author) as IEEE
                        % does not format its papers in such ways.

\usepackage{amsmath}  
\usepackage{lettrine}
\usepackage{titlesec}
\usepackage{graphics}
\usepackage{ragged2e}
\usepackage{multicol}
\usepackage{amsmath}
%\usepackage{titlesec} % From the American Mathematical Society
                        % A popular package that provides many helpful commands
                        % for dealing with mathematics. Note that the AMSmath
                        % package sets \interdisplaylinepenalty to 10000 thus
                        % preventing page breaks from occurring within multiline
                        % equations. Use:
%\interdisplaylinepenalty=2500
                        % after loading amsmath to restore such page breaks
                        % as IEEEtran.cls normally does. amsmath.sty is already
                        % installed on most LaTeX systems. The latest version
                        % and documentation can be obtained at:
                        % http://www.ctan.org/tex-archive/macros/latex/required/amslatex/math/



% Other popular packages for formatting tables and equations include:

%\usepackage{array}
% Frank Mittelbach's and David Carlisle's array.sty which improves the
% LaTeX2e array and tabular environments to provide better appearances and
% additional user controls. array.sty is already installed on most systems.
% The latest version and documentation can be obtained at:
% http://www.ctan.org/tex-archive/macros/latex/required/tools/

% V1.6 of IEEEtran contains the IEEEeqnarray family of commands that can
% be used to generate multiline equations as well as matrices, tables, etc.

% Also of notable interest:
% Scott Pakin's eqparbox package for creating (automatically sized) equal
% width boxes. Available:
% http://www.ctan.org/tex-archive/macros/latex/contrib/supported/eqparbox/

% *** Do not adjust lengths that control margins, column widths, etc. ***
% *** Do not use packages that alter fonts (such as pslatex).         ***
% There should be no need to do such things with IEEEtran.cls V1.6 and later.
%\titlespacing\section{0pt}{2pt plus 4pt minus 2pt}{2pt plus 2pt minus 2pt}
%\titlespacing\subsection{0pt}{2pt plus 4pt minus 2pt}{0pt plus 2pt minus 2pt}
%\titlespacing\subsubsection{0pt}{0pt plus 4pt minus 2pt}{0pt plus 2pt minus 2pt}

%\titlespacing\author{0pt}{0pt minus 12pt minus 8pt}{0pt minus 12pt minus 8pt}

%\titlespacing\abstract{0pt}{0pt minus 100pt minus 20pt}{0pt minus 40pt minus 40pt}

% Your document starts here!
\title{\LARGE {Software  Analysis Using Code Metrics}}
\author{\small {Darren Blanckensee | 1147279}}



\begin{document}

% Define document title and author
	
%	\thanks{Advisor: Dipl.--Ing.~Firstname Lastname, Lehrstuhl f\"ur Nachrichtentechnik, TUM, WS 2050/2051.
%}
	\markboth{University of Witwatersrand}{}
	\maketitle

% Write abstract here

\begin{abstract}
Due to the rising popularity in the area of Internet of Things (IoT) there has been significant research done on how to implement edge and fog computing in order to improve the speed and efficiency of any communication between edge devices, the fog gateway and the cloud if necessary. Edge and fog computing involve data transmission whether it is between edge devices, between edge devices and fog nodes or between fog gateways and the cloud. All of these forms of transmission can be made more efficient and faster using certain filtering and compression techniques. Using filtering and compression techniques will speed up response time and use less bandwidth which  will not only improve user experience but make the edge and fog computing processes more efficient in terms of space, time and energy. This would have direct effects on the efficiency of IoT systems. 
\end{abstract}


% Each section begins with a \section{title} command

\section{Introduction}
\noindent
\IEEEPARstart{S}{oftware} is being developed at an ever increasing rate. It is important when writing programs to follow good programming practices and to hold the code one writes to a certain standard so as to ensure a high quality product. Often when software projects are large with many different classes and files with thousands of lines to keep track of it becomes increasingly difficult to maintain a standard as it becomes impossible for developers to read all the lines of code and make sure that the quality requirements are met. One way to address this issue is to make use of software metrics. Software metrics are standards of measurement that provide developers with quantifiable measurements of various characteristics of the code they have developed. \\

\noindent
One of the most simple metrics often used as an example in software is the lines of code metric (LOC) which measures how many lines of code exist in a program. This metric alone does not however provide the developer with useful information relating to the quality of the software and whether or not good programming practices are being followed. As So and so once said Puthequoteherepleasedarren. Other more useful metrics exist and the use of these in analysing software projects is the purpose of this report. All code in this project (including the metrics tool used) is written in Python.  \\

\noindent
The Chosen Metrics section details which metrics have been selected to allow for sound analysis of a code base that provides developers with useful information that could be used to maintain a standard of coding. Along with the selected metrics, explanations of each metric will be provided so as to explain the importance of each of the chosen metrics and how they should be used. The Code Base Analysis section covers the analysis of the authors own code base along with three major releases of the open source software, Freevo Media Library. 



\section{Chosen Metrics}
\noindent
The metrics being used in this project are the following:
\begin{itemize}
\item Cyclomatic Complexity
\item Maintainability Index
\item Coverage
\item Halstead Metrics
\item Dependencies
\end{itemize}
\noindent

\section{Code Base Analysis}
\noindent

\section{Experimental Setup and Computational Model}
\noindent



\noindent
The code will be stored on a public repository on Github as will any papers or files relating to the project that are used to aid the authors. Anybody who wishes to validate the results of this project will then be able to download the source code and required files and will be able to test it on their own edge devices. 
\section{Preliminary Results}
\noindent
whoops

\section{Explanation of the Rest of the Work to be Accomplished}
\noindent
There are a number of tasks that still need to be done. Firstly the authors need to aqquire the smartphones and tablets that will be used as edge devices, root them and install ubuntu on them. Secondly the data that will be used to simulate the data that is produced by the sensors needs to be aqquired. Two types of data are needed data that is generated in realtime so as to mirror a device that monitors real time information such as weather stations that transmit data as it is generated and data that is generated and is only trasnmitted after a certain amount of time or after a certain amount of data has been collected for example weekly electrical load profiles for a smart house where data is collected throughout the week allof which is only transmitted at the end of each week.\\

%%%%%%%%%%%%%%%%%%%%%%%%%%%%%%
\noindent
After programming the Android devices via the SD cards, they are to be tested accordingly in order to determine whether they perform the required tasks. The authors will also research and implement the optimum techniques for accelerated computation.\\
%%%%%%%%%%%%%%%%%%%%%%%%%%%%%%

\noindent
More research into what various compression algortihms and filtering techniques exist in the areaa of edge and fog computing is to be done. The top performing algortihms and techniques are then to be implemented on the smartphones and tablets (edge devices). Actually using the compression and filtering techniques when transmitting the data between devices and the fog gateway should be at the discretion of the authors so that when testing takes place comparisons are made easy. Additionally, tests need to be developed and for this multiple types of queries have to be identified as the basis for the tests. At the time of writing the types of queries that are to be used are select queries (with conditions), mathematical select queries that return the sum or the aggregate of the data in the selection, join queries and possibly insert and delete queries. 

\section{Methods for Validations of Expected Results and Exceptions}
\noindent
At the time of writing this proposal it was unclear as to which languages and libraries would be used and therefore the most accurate timing method could not be determined. Once this information is known , research will be done to find the most accurate timing procedure. The time will be started when the query takes place and will stop when the result is returned. It is expected that when compression and filtering is implemented the time will be significantly shorter than when there is neither filtering nor compression. It is unclear how much faster it will be with the filtering and compression however that is part of the projects goal, to find out how much faster edge and fog computing queries can be run. \\

\noindent
Along with the time measurement there need also be another measurement that takes place to determine whether the query has returned the correct result. Accelerating the process of edge and fog computing is important however so is the integrity of the data as it is not helpful reducing the response time if this incurrs substantial errors. To validate the query results the result of the query when using compression and filtering will be compared with the result of the query when compression and filtering are not used. If identical then both results were succesfully generated and if not then a closer look will need to be taken at the data returned by the compression and filtering technique. It is important to note that bloom filters when used do, although not very often, return false positives. False positives would mean returning data that should not be there however without the filtering there would be significantly more data that need not be there so it is assumed that false positives are not fatal as overally the data will still have been significantly compressed and therefore the response time reduced.


\section{Risk Management}
\noindent
An immediate risk that is facing the project is damaging the devices used, which could possibly compromise the reliablilty of the data collected and/or used.\\

\section{Literature Review}
\noindent
According to \cite{accelerating}, there exist a number of compression algorithms in edge computing. Some of the more popular types of edge compression are various forms of time series compression whether it be sampling or representing a time period by the average of all the values within that time period \cite{time1}. This is a valid form of compression however it introduces a level of fuzziness as the values are not true values and the value at any given time can never be exact. Depending on the time period over which the average is calculated these averages'accuracy will vary. The more accurate the average the less compressed the data, this method therefore is not ideal. \\

\noindent
Another method suggested by \cite{PIP} is using perceptually important points to represent the data. This method also introduces fuzziness and like the previous method depending on the number of perceptually important points the representations accuracy varies in a manner that is inversely proportional to the compression rate.  Furthermore these compression methods are suited for data sets as opposed to data streams as stated in \cite{accelerating}. \\

bbbbbbeeeeeeeeeeefffffffffffooooooooooorrrrrrrrrreeeeeeeeeeee

\noindent
The suggested method for this project is the use of a Bloom filter which is a way to  test if a piece of data exists in a set or not. 




\section{Schedule and Time-line}
\noindent


\section{Summary of Proposal and Planned Additional Work to Complete}
\noindent




% \begin{figure}[H]
% \centering 
% \includegraphics[width=\columnwidth]{warehouse1}
% \centering 
%  \caption {Lighting design for Sasol wax warehouse storage area using first set of lights }
% \end{figure}




\begin{thebibliography}{5}


\bibitem{accelerating} Apostolos Papageorgiou, Bin Cheng, Erno Kovacs, Real-Time Data Reduction at the Network Edge of Internet-of-Things Systems. NEC Laboratories Europe Heidelberg, Germany, 2015.

\bibitem{time1}
B.-K. Yi and C. Faloutsos. Fast Time Sequence Indexing for Arbitrary
Lp Norms. In Proceedings of the 26th International Conference on
Very Large Data Bases, VLDB ’00, pages 385–394. Morgan Kaufmann
Publishers Inc., 2000.

\bibitem{PIP}
F. Chung, T. Fu, R. Luk, and V. Ng. Flexible time series pattern
matching based on Perceptually Important Points. In International
Joint Conference on Artificial Intelligence, Workshop on Learning from
Temporal and Spatial Data, pages 1–7, 2001.


\bibitem{PIP}
Weisong Shi, Fellow, IEEE, Jie Cao, Student Member, IEEE, Quan Zhang, Student Member, IEEE, Youhuizi Li, and Lanyu Xu. Edge Computing: Vision and Challenges. 





% \bibitem{c2} W.-K. Chen, Linear Networks and Systems (Book style).	Belmont, CA: Wadsworth, 1993, pp. 123Ð135.
% \bibitem{c3} H. Poor, An Introduction to Signal Detection and Estimation.   New York: Springer-Verlag, 1985, ch. 4.
% \bibitem{c4} B. Smith, ÒAn approach to graphs of linear forms (Unpublished work style),Ó unpublished.
% \bibitem{c5} E. H. Miller, ÒA note on reflector arrays (Periodical styleÑAccepted for publication),Ó IEEE Trans. Antennas Propagat., to be publised.
% \bibitem{c6} J. Wang, ÒFundamentals of erbium-doped fiber amplifiers arrays (Periodical styleÑSubmitted for publication),Ó IEEE J. Quantum Electron., submitted for publication.
% \bibitem{c7} C. J. Kaufman, Rocky Mountain Research Lab., Boulder, CO, private communication, May 1995.
% \bibitem{c8} Y. Yorozu, M. Hirano, K. Oka, and Y. Tagawa, ÒElectron spectroscopy studies on magneto-optical media and plastic substrate interfaces(Translation Journals style),Ó IEEE Transl. J. Magn.Jpn., vol. 2, Aug. 1987, pp. 740Ð741 [Dig. 9th Annu. Conf. Magnetics Japan, 1982, p. 301].
% \bibitem{c9} M. Young, The Techincal Writers Handbook.  Mill Valley, CA: University Science, 1989.
% \bibitem{c10} J. U. Duncombe, ÒInfrared navigationÑPart I: An assessment of feasibility (Periodical style),Ó IEEE Trans. Electron Devices, vol. ED-11, pp. 34Ð39, Jan. 1959.
% \bibitem{c11} S. Chen, B. Mulgrew, and P. M. Grant, ÒA clustering technique for digital communications channel equalization using radial basis function networks,Ó IEEE Trans. Neural Networks, vol. 4, pp. 570Ð578, July 1993.
% \bibitem{c12} R. W. Lucky, ÒAutomatic equalization for digital communication,Ó Bell Syst. Tech. J., vol. 44, no. 4, pp. 547Ð588, Apr. 1965.
% \bibitem{c13} S. P. Bingulac, ÒOn the compatibility of adaptive controllers (Published Conference Proceedings style),Ó in Proc. 4th Annu. Allerton Conf. Circuits and Systems Theory, New York, 1994, pp. 8Ð16.
% \bibitem{c14} G. R. Faulhaber, ÒDesign of service systems with priority reservation,Ó in Conf. Rec. 1995 IEEE Int. Conf. Communications, pp. 3Ð8.
% \bibitem{c15} W. D. Doyle, ÒMagnetization reversal in films with biaxial anisotropy,Ó in 1987 Proc. INTERMAG Conf., pp. 2.2-1Ð2.2-6.
% \bibitem{c16} G. W. Juette and L. E. Zeffanella, ÒRadio noise currents n short sections on bundle conductors (Presented Conference Paper style),Ó presented at the IEEE Summer power Meeting, Dallas, TX, June 22Ð27, 1990, Paper 90 SM 690-0 PWRS.
% \bibitem{c17} J. G. Kreifeldt, ÒAn analysis of surface-detected EMG as an amplitude-modulated noise,Ó presented at the 1989 Int. Conf. Medicine and Biological Engineering, Chicago, IL.
% \bibitem{c18} J. Williams, ÒNarrow-band analyzer (Thesis or Dissertation style),Ó Ph.D. dissertation, Dept. Elect. Eng., Harvard Univ., Cambridge, MA, 1993. 
% \bibitem{c19} N. Kawasaki, ÒParametric study of thermal and chemical nonequilibrium nozzle flow,Ó M.S. thesis, Dept. Electron. Eng., Osaka Univ., Osaka, Japan, 1993.
% \bibitem{c20} J. P. Wilkinson, ÒNonlinear resonant circuit devices (Patent style),Ó U.S. Patent 3 624 12, July 16, 1990. 





\end{thebibliography}

	% This is how you define a table: the [!hbt] means that LaTeX is forced (by the !) to place the table exactly here (by h), or if that doesnt work because of a pagebreak or so, it tries to place the table to the bottom of the page (by b) or the top (by t).


	% If you have questions about how to write mathematical formulas in LaTeX, please read a LaTeX book or the 'Not So Short Introduction to LaTeX': tobi.oetiker.ch/lshort/lshort.pdf

% Now we need a bibliography:


% Your document ends here!
\end{document}