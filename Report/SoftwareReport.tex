\documentclass[11pt,journal, a4paper]{IEEEtran}

\makeatletter
\newcommand\subparagraph{%
  \@startsection{subparagraph}{5}
  {\parindent}
  {3.25ex \@plus 1ex \@minus .2ex}
  {-1em}
  {\normalfont\normalsize\bfseries}}
\makeatother
\usepackage{titlesec}
\let\subparagraph\relax % You don't want to use \subparagraph

% some very useful LaTeX packages include:

%\usepackage{cite}      % Written by Donald Arseneau
                        % V1.6 and later of IEEEtran pre-defines the format
                        % of the cite.sty package \cite{} output to follow
                        % that of IEEE. Loading the cite package will
                        % result in citation numbers being automatically
                        % sorted and properly "ranged". i.e.,
                        % [1], [9], [2], [7], [5], [6]
                        % (without using cite.sty)
                        % will become:
                        % [1], [2], [5]--[7], [9] (using cite.sty)
                        % cite.sty's \cite will automatically add leading
                        % space, if needed. Use cite.sty's noadjust option
                        % (cite.sty V3.8 and later) if you want to turn this
                        % off. cite.sty is already installed on most LaTeX
                        % systems. The latest version can be obtained at:
                        % http://www.ctan.org/tex-archive/macros/latex/contrib/supported/cite/
\usepackage{float}
\usepackage{graphicx}   % Written by David Carlisle and Sebastian Rahtz
                        % Required if you want graphics, photos, etc.
                        % graphicx.sty is already installed on most LaTeX
                        % systems. The latest version and documentation can
                        % be obtained at:
                        % http://www.ctan.org/tex-archive/macros/latex/required/graphics/
                        % Another good source of documentation is "Using
                        % Imported Graphics in LaTeX2e" by Keith Reckdahl
                        % which can be found as esplatex.ps and epslatex.pdf
                        % at: http://www.ctan.org/tex-archive/info/


%\usepackage{psfrag}    % Written by Craig Barratt, Michael C. Grant,
                        % and David Carlisle
                        % This package allows you to substitute LaTeX
                        % commands for text in imported EPS graphic files.
                        % In this way, LaTeX symbols can be placed into
                        % graphics that have been generated by other
                        % applications. You must use latex->dvips->ps2pdf
                        % workflow (not direct pdf output from pdflatex) if
                        % you wish to use this capability because it works
                        % via some PostScript tricks. Alternatively, the
                        % graphics could be processed as separate files via
                        % psfrag and dvips, then converted to PDF for
                        % inclusion in the main file which uses pdflatex.
                        % Docs are in "The PSfrag System" by Michael C. Grant
                        % and David Carlisle. There is also some information
                        % about using psfrag in "Using Imported Graphics in
                        % LaTeX2e" by Keith Reckdahl which documents the
                        % graphicx package (see above). The psfrag package
                        % and documentation can be obtained at:
                        % http://www.ctan.org/tex-archive/macros/latex/contrib/supported/psfrag/

%\usepackage{subfigure} % Written by Steven Douglas Cochran
                        % This package makes it easy to put subfigures
                        % in your figures. i.e., "figure 1a and 1b"
                        % Docs are in "Using Imported Graphics in LaTeX2e"
                        % by Keith Reckdahl which also documents the graphicx
                        % package (see above). subfigure.sty is already
                        % installed on most LaTeX systems. The latest version
                        % and documentation can be obtained at:
                        % http://www.ctan.org/tex-archive/macros/latex/contrib/supported/subfigure/

\usepackage{url}        % Written by Donald Arseneau
                        % Provides better support for handling and breaking
                        % URLs. url.sty is already installed on most LaTeX
                        % systems. The latest version can be obtained at:
                        % http://www.ctan.org/tex-archive/macros/latex/contrib/other/misc/
                        % Read the url.sty source comments for usage information.

%\usepackage{stfloats}  % Written by Sigitas Tolusis
                        % Gives LaTeX2e the ability to do double column
                        % floats at the bottom of the page as well as the top.
                        % (e.g., "\begin{figure*}[!b]" is not normally
                        % possible in LaTeX2e). This is an invasive package
                        % which rewrites many portions of the LaTeX2e output
                        % routines. It may not work with other packages that
                        % modify the LaTeX2e output routine and/or with other
                        % versions of LaTeX. The latest version and
                        % documentation can be obtained at:
                        % http://www.ctan.org/tex-archive/macros/latex/contrib/supported/sttools/
                        % Documentation is contained in the stfloats.sty
                        % comments as well as in the presfull.pdf file.
                        % Do not use the stfloats baselinefloat ability as
                        % IEEE does not allow \baselineskip to stretch.
                        % Authors submitting work to the IEEE should note
                        % that IEEE rarely uses double column equations and
                        % that authors should try to avoid such use.
                        % Do not be tempted to use the cuted.sty or
                        % midfloat.sty package (by the same author) as IEEE
                        % does not format its papers in such ways.



\usepackage{amsmath}  
\usepackage{lettrine}
\usepackage{titlesec}
\usepackage{graphics}
\usepackage{ragged2e}
\usepackage{multicol}
\usepackage[linesnumbered,ruled]{algorithm2e}
%\usepackage{titlesec} % From the American Mathematical Society
                        % A popular package that provides many helpful commands
                        % for dealing with mathematics. Note that the AMSmath
                        % package sets \interdisplaylinepenalty to 10000 thus
                        % preventing page breaks from occurring within multiline
                        % equations. Use:
%\interdisplaylinepenalty=2500
                        % after loading amsmath to restore such page breaks
                        % as IEEEtran.cls normally does. amsmath.sty is already
                        % installed on most LaTeX systems. The latest version
                        % and documentation can be obtained at:
                        % http://www.ctan.org/tex-archive/macros/latex/required/amslatex/math/
\usepackage{listings}


% Other popular packages for formatting tables and equations include:

%\usepackage{array}
% Frank Mittelbach's and David Carlisle's array.sty which improves the
% LaTeX2e array and tabular environments to provide better appearances and
% additional user controls. array.sty is already installed on most systems.
% The latest version and documentation can be obtained at:
% http://www.ctan.org/tex-archive/macros/latex/required/tools/

% V1.6 of IEEEtran contains the IEEEeqnarray family of commands that can
% be used to generate multiline equations as well as matrices, tables, etc.

% Also of notable interest:
% Scott Pakin's eqparbox package for creating (automatically sized) equal
% width boxes. Available:
% http://www.ctan.org/tex-archive/macros/latex/contrib/supported/eqparbox/

% *** Do not adjust lengths that control margins, column widths, etc. ***
% *** Do not use packages that alter fonts (such as pslatex).         ***
% There should be no need to do such things with IEEEtran.cls V1.6 and later.
%\titlespacing\section{0pt}{2pt plus 4pt minus 2pt}{2pt plus 2pt minus 2pt}
%\titlespacing\subsection{0pt}{2pt plus 4pt minus 2pt}{0pt plus 2pt minus 2pt}
%\titlespacing\subsubsection{0pt}{0pt plus 4pt minus 2pt}{0pt plus 2pt minus 2pt}

%\titlespacing\author{0pt}{0pt minus 12pt minus 8pt}{0pt minus 12pt minus 8pt}

%\titlespacing\abstract{0pt}{0pt minus 100pt minus 20pt}{0pt minus 40pt minus 40pt}

% Your document starts here!
\title{\LARGE {Software  Analysis Using Code Metrics}}
\author{\small {Darren Blanckensee | 1147279}}



\begin{document}

% Define document title and author
	
%	\thanks{Advisor: Dipl.--Ing.~Firstname Lastname, Lehrstuhl f\"ur Nachrichtentechnik, TUM, WS 2050/2051.
%}
	\markboth{University of Witwatersrand}{}
	\maketitle

% Write abstract here

\begin{abstract}
Due to the rising popularity in the area of Internet of Things (IoT) there has been significant research done on how to implement edge and fog computing in order to improve the speed and efficiency of any communication between edge devices, the fog gateway and the cloud if necessary. Edge and fog computing involve data transmission whether it is between edge devices, between edge devices and fog nodes or between fog gateways and the cloud. All of these forms of transmission can be made more efficient and faster using certain filtering and compression techniques. Using filtering and compression techniques will speed up response time and use less bandwidth which  will not only improve user experience but make the edge and fog computing processes more efficient in terms of space, time and energy. This would have direct effects on the efficiency of IoT systems. 
\end{abstract}


% Each section begins with a \section{title} command

\section{Introduction}
\noindent
\IEEEPARstart{S}{oftware} is being developed at an ever increasing rate. It is important when writing programs to follow good programming practices and to hold the code one writes to a certain standard so as to ensure a high quality product. Often when software projects are large with many different classes and files with thousands of lines to keep track of it becomes increasingly difficult to maintain a standard as it becomes impossible for developers to read all the lines of code and make sure that the quality requirements are met. One way to address this issue is to make use of software metrics. Software metrics are standards of measurement that provide developers with quantifiable measurements of various characteristics of the code they have developed. \\

\noindent
One of the most simple metrics often used as an example in software is the lines of code metric (LOC) which measures how many lines of code exist in a program. This metric alone does not however provide the developer with useful information relating to the quality of the software and whether or not good programming practices are being followed. As So and so once said Puthequoteherepleasedarren. Other more useful metrics exist and the use of these in analysing software projects is the purpose of this report. All code in this project (including the metrics tool used) is written in Python.  \\

\noindent
The Chosen Metrics section details which metrics have been selected to allow for sound analysis of a code base that provides developers with useful information that could be used to maintain a standard of coding. Along with the selected metrics, explanations of each metric will be provided so as to explain the importance of each of the chosen metrics and how they should be used. The Code Base Analysis section covers the analysis of the authors own code base along with three major releases of the open source software, Freevo Media Library. 



\section{Chosen Metrics}
\noindent
The metrics being used in this project are the following:
\begin{itemize}
\item Cyclomatic Complexity
\item Maintainability Index
\item Coverage
\item Halstead Metrics
\item Dependencies
\end{itemize}
\noindent
The tools used to calculate these metrics are Radon, Pylint, Graphviz, Webdot and Snakefood. Each of these metrics used are explained in the subsections below.

\subsection{Cyclomatic Complexity}
\noindent
The cyclomatic complexity metric, calculated using the Radon tool, is used to determine the number of distinct paths a program can take during running. It gives the developer an idea of how many decisions are being made by their code. There are various constructs and conditional statements or decisions that have an effect on the cyclomatic complexity of any program. These and their additive  effects on the cyclomatic complexity are shown in the table below. 

\begin{table}[H]
\centering
\caption{Statements and their effects on cyclomatic complexity}
\label{my-label}
\begin{tabular}{|l|l|}
\hline
Statement         & Effect On Cyclomatic Complexity \\ \hline
If                & 1                               \\ \hline
Elif              & 1                               \\ \hline
Else              & 0                               \\ \hline
For               & 1                               \\ \hline
While             & 1                               \\ \hline
Except            & 1                               \\ \hline
Finally           & 0                               \\ \hline
With              & 1                               \\ \hline
Assert            & 1                               \\ \hline
Comprehension     & 1                               \\ \hline
Boolean Operators & 1                               \\ \hline
\end{tabular}
\end{table}

\noindent
An example (not the author's code base) and its cyclomatic complexity are shown in figure 1 and the paragraph that follows respectively. 


\lstinputlisting[caption=Example code to explain cyclomatic complexity metric.,language=python]{dog.py}

\noindent
While this code is simple and may not follow the best programming practices it is sufficient to explain how the cyclomatic complexity metric works. There are four ways this code can run. The first being if the dog is hungry and it is home then the dog will be fed. The second being if the dog is hungry but is not home then the dog will be located and fed. The third way is if the dog is not hungry and is not home then the dog will be located. The last way is if the dog is not hungry but is home then nothing is to be done. Because there are four ways this program can be run the cyclomatic complexity of this code is 4????????.\\  



\noindent
Radon was used to calculate the cyclomatic complexity of the various code bases analysed in this project. Radon's cyclomatic complexity measurement outputs a letter between A and F along with a score larger than zero relating to the number of decisions. A relates to a section of code that has less than five decisions while F relates to more than 41 decisions. \\

\noindent
The cyclomatic complexity command on Radon returns these letters and scores for each class, method and function in a section of code. Code with many decisions means many different ways the code can run which leads to high risk that code may not behave as the developer expects or intends. It is ideal to have a cyclomatic complexity score of less than 10 or B \cite{mcCabe}.  


\subsection{Maintainability Index}
\noindent
The maintainability index, calculated using the Radon tool, is used to determine how easily a section of code can be maintained. This metric measures how easy it would be to change this code in future. The maintainability index as a metric was introduced is calculated using the equation below. 


\begin{equation}
\resizebox{0.5\textwidth}{!}{ $MI = \emph{max} [0,100*\frac{171-5.2\ln\emph{V}-0.23\emph{G} -16.2\ln\emph{L} + 50\sin(\sqrt{2.4\emph{C}}) }{171}]$}
\end{equation}
Where V is the Halstead volume, G is cyclomatic complexity, L is the number of source lines of code and C is the percent of comment lines in the program converted to radians.


\section{Code Base Analysis}
\noindent

\section{Experimental Setup and Computational Model}
\noindent


\section{Preliminary Results}
\noindent

\section{Explanation of the Rest of the Work to be Accomplished}
\noindent


\section{Methods for Validations of Expected Results and Exceptions}
\noindent


\section{Risk Management}
\noindent


\section{Literature Review}
\noindent



\section{Schedule and Time-line}
\noindent


\section{Summary of Proposal and Planned Additional Work to Complete}
\noindent




% \begin{figure}[H]
% \centering 
% \includegraphics[width=\columnwidth]{warehouse1}
% \centering 
%  \caption {Lighting design for Sasol wax warehouse storage area using first set of lights }
% \end{figure}




\begin{thebibliography}{5}


\bibitem{mcCabe} Apostolos Papageorgiou, Bin Cheng, Erno Kovacs, Real-Time Data Reduction at the Network Edge of Internet-of-Things Systems. NEC Laboratories Europe Heidelberg, Germany, 2015.






% \bibitem{c2} W.-K. Chen, Linear Networks and Systems (Book style).	Belmont, CA: Wadsworth, 1993, pp. 123Ð135.
% \bibitem{c3} H. Poor, An Introduction to Signal Detection and Estimation.   New York: Springer-Verlag, 1985, ch. 4.
% \bibitem{c4} B. Smith, ÒAn approach to graphs of linear forms (Unpublished work style),Ó unpublished.
% \bibitem{c5} E. H. Miller, ÒA note on reflector arrays (Periodical styleÑAccepted for publication),Ó IEEE Trans. Antennas Propagat., to be publised.
% \bibitem{c6} J. Wang, ÒFundamentals of erbium-doped fiber amplifiers arrays (Periodical styleÑSubmitted for publication),Ó IEEE J. Quantum Electron., submitted for publication.
% \bibitem{c7} C. J. Kaufman, Rocky Mountain Research Lab., Boulder, CO, private communication, May 1995.
% \bibitem{c8} Y. Yorozu, M. Hirano, K. Oka, and Y. Tagawa, ÒElectron spectroscopy studies on magneto-optical media and plastic substrate interfaces(Translation Journals style),Ó IEEE Transl. J. Magn.Jpn., vol. 2, Aug. 1987, pp. 740Ð741 [Dig. 9th Annu. Conf. Magnetics Japan, 1982, p. 301].
% \bibitem{c9} M. Young, The Techincal Writers Handbook.  Mill Valley, CA: University Science, 1989.
% \bibitem{c10} J. U. Duncombe, ÒInfrared navigationÑPart I: An assessment of feasibility (Periodical style),Ó IEEE Trans. Electron Devices, vol. ED-11, pp. 34Ð39, Jan. 1959.
% \bibitem{c11} S. Chen, B. Mulgrew, and P. M. Grant, ÒA clustering technique for digital communications channel equalization using radial basis function networks,Ó IEEE Trans. Neural Networks, vol. 4, pp. 570Ð578, July 1993.
% \bibitem{c12} R. W. Lucky, ÒAutomatic equalization for digital communication,Ó Bell Syst. Tech. J., vol. 44, no. 4, pp. 547Ð588, Apr. 1965.
% \bibitem{c13} S. P. Bingulac, ÒOn the compatibility of adaptive controllers (Published Conference Proceedings style),Ó in Proc. 4th Annu. Allerton Conf. Circuits and Systems Theory, New York, 1994, pp. 8Ð16.
% \bibitem{c14} G. R. Faulhaber, ÒDesign of service systems with priority reservation,Ó in Conf. Rec. 1995 IEEE Int. Conf. Communications, pp. 3Ð8.
% \bibitem{c15} W. D. Doyle, ÒMagnetization reversal in films with biaxial anisotropy,Ó in 1987 Proc. INTERMAG Conf., pp. 2.2-1Ð2.2-6.
% \bibitem{c16} G. W. Juette and L. E. Zeffanella, ÒRadio noise currents n short sections on bundle conductors (Presented Conference Paper style),Ó presented at the IEEE Summer power Meeting, Dallas, TX, June 22Ð27, 1990, Paper 90 SM 690-0 PWRS.
% \bibitem{c17} J. G. Kreifeldt, ÒAn analysis of surface-detected EMG as an amplitude-modulated noise,Ó presented at the 1989 Int. Conf. Medicine and Biological Engineering, Chicago, IL.
% \bibitem{c18} J. Williams, ÒNarrow-band analyzer (Thesis or Dissertation style),Ó Ph.D. dissertation, Dept. Elect. Eng., Harvard Univ., Cambridge, MA, 1993. 
% \bibitem{c19} N. Kawasaki, ÒParametric study of thermal and chemical nonequilibrium nozzle flow,Ó M.S. thesis, Dept. Electron. Eng., Osaka Univ., Osaka, Japan, 1993.
% \bibitem{c20} J. P. Wilkinson, ÒNonlinear resonant circuit devices (Patent style),Ó U.S. Patent 3 624 12, July 16, 1990. 





\end{thebibliography}

	% This is how you define a table: the [!hbt] means that LaTeX is forced (by the !) to place the table exactly here (by h), or if that doesnt work because of a pagebreak or so, it tries to place the table to the bottom of the page (by b) or the top (by t).


	% If you have questions about how to write mathematical formulas in LaTeX, please read a LaTeX book or the 'Not So Short Introduction to LaTeX': tobi.oetiker.ch/lshort/lshort.pdf

% Now we need a bibliography:


% Your document ends here!
\end{document}